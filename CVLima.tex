%% start of file `template.tex'.
%% Copyright 2006-2013 Xavier Danaux (xdanaux@gmail.com).
%
% This work may be distributed and/or modified under the
% conditions of the LaTeX Project Public License version 1.3c,
% available at http://www.latex-project.org/lppl/.


\documentclass[11pt,sans]{moderncv}        % possible options include font size ('10pt', '11pt' and '12pt'), paper size ('a4paper', 'letterpaper', 'a5paper', 'legalpaper', 'executivepaper' and 'landscape') and font family ('sans' and 'roman')


% moderncv themes
\moderncvstyle{mybanking}                             % style options are 'casual' (default), 'classic', 'oldstyle' and 'banking'
\moderncvcolor{myweb}                               % color options 'blue' (default), 'orange', 'green', 'red', 'purple', 'grey' and 'black'
%\renewcommand{\familydefault}{\sfdefault}         % to set the default font; use '\sfdefault' for the default sans serif font, '\rmdefault' for the default roman one, or any tex font name
\nopagenumbers{}                                  % uncomment to suppress automatic page numbering for CVs longer than one page -- Also uncomment line 106 through 111

% character encoding
\usepackage[utf8]{inputenc}                       % if you are not using xelatex ou lualatex, replace by the encoding you are using
%\usepackage{CJKutf8}                              % if you need to use CJK to typeset your resume in Chinese, Japanese or Korean

% adjust the page margins
\usepackage[scale=0.7]{geometry}
%\setlength{\hintscolumnwidth}{3cm}                % if you want to change the width of the column with the dates
%\setlength{\makecvtitlenamewidth}{10cm}           % for the 'classic' style, if you want to force the width allocated to your name and avoid line breaks. be careful though, the length is normally calculated to avoid any overlap with your personal info; use this at your own typographical risks...
	\geometry{
	paper=a4paper, % Change to letterpaper for US letter OR a4paper 
  %top=3.32cm, % Top margin for A4paper
  top=3.38cm, % Top margin for LETTERpaper
	%bottom=3.32cm, % Bottom margin for A4paper
  bottom=3.38cm, % Bottom margin for LETTERpaper
	%left=3.27cm, % Left margin for A4paper
  left=3.27cm, % Left margin for LETTERpaper
	%right=3.27cm, % Right margin for A4paper
  right=3.27cm, % Right margin for LETTERpaper
	%showframe, % Uncomment to show how the type block is set on the page
}
\usepackage{graphicx}
\usepackage{textcomp}
\usepackage{fontawesome}

% personal data
\name{Juliana}{Faccio Lima}
%\title{CV}                               % optional, remove / comment the line if not wanted
\address{Rua Visconde do Rio Branco, 117, Bairro Jabaquara}{Paraty, Rio de Janeiro, Brazil, 23970-000}% optional, remove / comment the line if not wanted; the "postcode city" and and "country" arguments can be omitted or provided empty
\phone[mobile]{+55~(24)~9~8164~1512}                   % optional, remove / comment the line if not wanted
\phone[mobile]{+1~(805)~617~3173}                    % optional, remove / comment the line if not wanted
%\phone[fax]{+3~(456)~789~012}                      % optional, remove / comment the line if not wanted
\email{jfacciolima@gmail.com}                               % optional, remove / comment the line if not wanted
\homepage{julianaflima.github.io/}                         % optional, remove / comment the line if not wanted
%\extrainfo{additional information}                 % optional, remove / comment the line if not wanted
%\photo[64pt][0.4pt]{picture}                       % optional, remove / comment the line if not wanted; '64pt' is the height the picture must be resized to, 0.4pt is the thickness of the frame around it (put it to 0pt for no frame) and 'picture' is the name of the picture file
%\quote{Some quote}                                 % optional, remove / comment the line if not wanted

% to show numerical labels in the bibliography (default is to show no labels); only useful if you make citations in your resume
%\makeatletter
%\renewcommand*{\bibliographyitemlabel}{\@biblabel{\arabic{enumiv}}}
%\makeatother
%\renewcommand*{\bibliographyitemlabel}{[\arabic{enumiv}]}% CONSIDER REPLACING THE ABOVE BY THIS

% bibliography with mutiple entries
%\usepackage{multibib}
%\newcites{book,misc}{{Books},{Others}}
%----------------------------------------------------------------------------------
%            content
%----------------------------------------------------------------------------------
\begin{document}
%\begin{CJK*}{UTF8}{gbsn}                          % to typeset your resume in Chinese using CJK
%-----       resume       ---------------------------------------------------------
\makecvtitle

%\vspace{.5cm} 
\section{Employment}
\cventry{Spring 2020, 2021}{Visiting Assistant Professor}{Ashoka University}{}{}{}
\vspace{1mm}

\section{Education}
\cventry{2012--2018}{Ph.D. Philosophy}{University of California at Santa Barbara}
{} % City
{} % \textit{Grade}
{}  % arguments 3 to 6 can be left empty
\vspace{.5mm}

\vspace{1mm}
\cventry{2012--2015}{M.A. Philosophy}{UC Santa Barbara}
{} %City
{} %\textit{Grade}
{} %Description

\vspace{1mm}
\cventry{2008--2010}{M.A. Logic and Metaphysics}{Federal University of Rio de Janeiro}
{} %City
{} %\textit{Grade}
{} %Description

\vspace{1mm}
\cventry{2003--2007}{B.A. Philosophy}{Federal University of Paraná}
{} %City
{} %\textit{Grade}
{} %Description
\vspace{1mm}

\section{Research Interests}
\cvitem{AOS}{Philosophy of Language, Philosophy of Mind.}
\cvitem{AOC}{Action Theory, Epistemology, Logic.}
%\vspace{1mm}



%%%%%%%%%%%%%%%%%%%%%%%%%
% Comment out for nopagenumbering
%\pagebreak
%\pagestyle{fancyplain}
%\fancyhf{} % sets both header and footer to nothing
%\renewcommand{\headrulewidth}{0pt} 
%\lfoot{\small J.F.Lima -- CV}
%\fancyfoot[R]{\parbox[b]{\pagenumberwidth}{\color{color2}\pagenumberfont\strut\thepage/\pageref{lastpage}}}
%%%%%%%%%%%%%%%%%%%%

%\pagebreak
\section{Publications}

\cvitem{forthcoming}{`How Can Millians Believe in Superheroes?', \emph{Organon F}.}
%\vspace{1mm}

\cvitem{2018}{`Indexicality and Action: Why We Need Indexical Beliefs to Motivate Intentional Actions,' \emph{Inquiry}, available online: \url{https://www.tandfonline.com/doi/abs/10.1080/0020174X.2018.1487881?journalCode=sinq20}.}
%\vspace{1mm}

\cvitem{2016}{`Frege's Puzzle' (original title: `O Enigma de Frege'), \emph{Revista Fundamento}, 1: 39--73.}
%\vspace{1mm}

\cvitem{2010}{`Can Demonstratives Not Have Senses?' (original title: `Podem Demonstrativos Não Ter Sentido?'), \emph{Ítaca}, 15: 382--93.}
%\vspace{1mm}


\cvitem{2009}{`Unarticulated Constituents: A Reply to Cappelen and Lepore' (original title: `Constituintes Inarticulados: Uma Resposta a Cappelen e Lepore'), \emph{Ítaca}, 11: 145--50.}
\vspace{2mm}


\subsection{Book Reviews}
\cvitem{2014}{Herman Cappelen, Josh Dever, \emph{The Inessential Indexical}, \emph{Polish Journal of Philosophy}, 8: 77--80.}
\vspace{2mm}

\subsection{Work in Progress}

\cvitem{}{\emph{Philosophy of Language} (original title: \emph{Filosofia da Linguagem}), ed. \& translators Juliana Lima \& Sagid Salles, online: NEPFil online, hard copy: Editora da Universidade Federal de Pelotas (book to be published in 2023)}
\vspace{1mm}

\cvitem{}{`Semantic Content, Belief-Content, and Belief Ascription' (under review)}
\vspace{1mm}

\cvitem{}{'We-thoughts and Joint Indexical Thoughts'}
\vspace{1mm}

\cvitem{}{`Semantic and Assertoric Content'}
\vspace{1mm}

\cvitem{}{`Mates's Problem'}
\vspace{1mm}


%\cvitem{}{`The Myth of the Selfish Indexical'}
%\vspace{1mm}

\cvitem{}{`The Extended Mind and Google: Should Instructors Encourage Students to Use Internet Resources?'}
\vspace{1mm}



\section{Presentations}
%\cvitem{}{*: Paper/abstract accepted but not attended.}
%\vspace{2mm} 

\cvitem{2021 forthcoming}{`What-Is-Said by Belief Ascriptions', 3rd Context, Cognition and Communication Conference, University of Warsaw, Poland.}
\vspace{1mm}

\cvitem{2020}{`Group Action and Indexical Beliefs', Departmental Talk, Ashoka University, India.}
\vspace{1mm}

\cvitem{2020}{`A Puzzle About Understanding', Philosophical Society of Finland Colloquium, Helsinki, Finland.}
\vspace{1mm}

\cvitem{2019}{`Understanding What-She-Said', 71st Annual Northwest Philosophy Conference, Pacific University, US.}
\vspace{1mm}

\cvitem{2019}{Commenting on Norris, J., `Behavioristic vs. Intuitive Rational Choice Theory', 71st Annual Northwest Philosophy Conference, Pacific University, US.}
\vspace{1mm}

\cvitem{2019}{`Joint Indexical Beliefs and Group Agency', New Horizons in Action and Agency, University of Helsinki, Finland.}
\vspace{1mm}

\cvitem{2019}{`Action-Based Indexicality', Second Bochum Early Career Workshop in Philosophy of Mind and of Cognitive Science, Ruhr-Universität Bochum, Germany.}
\vspace{1mm}

\cvitem{2019}{Commenting on Venter, E. `What Indexicality Tells Us About Self-Representation', Second Bochum Early Career Workshop in Philosophy of Mind and of Cognitive Science, Ruhr-Universität Bochum, Germany.}
\vspace{1mm}


\cvitem{2019}{`What-Is-Said by Belief Ascriptions', Issues in Contemporary Semantics and Ontology (ICSO) V, Argentinian Society of Philosophical Analysis, University of Buenos Aires, Argentina.}
\vspace{1mm}

%\newpage % for A4paper
\cvitem{2019}{'On Ascribing Beliefs and Intentionality', Italian Association of Cognitive Science (AISC) Midterm Conference, IMT School for Advanced Studies Lucca, Italy}
\vspace{1mm}

\cvitem{2019}{`Multiple Iterations of Attitude Verbs', International Conference on Philosophy of Language and Linguistics, University of Lodz, Poland.}
\vspace{1mm}

\cvitem{2019}{`Semantic and Assertoric Content', Philosophy Research Forum, University of Miami, US.}
\vspace{1mm}

\cvitem{2018}{`Instructor Diversity in Student Evaluations', So-Cal MAP Workshop, UC Los Angeles, US. Co-presenter: Sherri Lynn Conklin.}
\vspace{1mm}

\pagebreak
\cvitem{2017}{`Thinking (Really) Selflessly', Workshop: What is the Problem of First-Person Thought?, University of Barcelona, Spain.}
\vspace{1mm}

\cvitem{2017}{`Thinking Selflessly', Graduate Philosophy Colloquium, UC Santa Barbara, US.}
\vspace{1mm}

\cvitem{2015}{`Neither \emph{De Dicto} nor \emph{De Re}: Indexical Beliefs are \emph{Sui Generis}', Federal University of Paraná, Brazil.}
\vspace{1mm}

%\pagebreak
\cvitem{2010}{`Can Demonstratives Not Have Sense?' (original title: `Podem Demonstrativos Não Ter Sentido?'), Encontro National de Filosofia da Associação Nacional de Pós-Graduação em Filosofia, Brazil.}
\vspace{1mm}

\cvitem{2010}{`Can Demonstratives Have Sense?' (original title: `Podem Demonstrativos Ter Sentido?'), Seminário de Pesquisa, Federal University of Rio de Janeiro, Brazil.}
\vspace{1mm}

\cvitem{2009}{`The Semantic of Indexicals' (original title: `A Semântica dos Indexicais'), Portuguese Society for Analytic Philosophy, Portugal.}
\vspace{1mm}

\cvitem{2009}{`The Semantic of Indexicals' (original title: `Semântica dos Indexicais'), Graduate Philosophy Conference, Federal University of Rio de Janeiro, Brazil.}
\vspace{1mm}

\cvitem{2008}{`Binding Argument: A Reply to Cappelen and Lepore' (original title: Binding Argument: Uma Resposta a Cappelen e Lepore), Graduate Philosophy Conference, Federal University of Rio de Janeiro, Brazil.}
%\vspace{1mm}

\section{Research Activity \& Grants}
\cvitem{2012}{Project Philosophy -- Developing a Social Aspect of Undergraduate Education (original title: ``Projeto Filosofia -- Melhoria da Qualidade Social da Graduação''), Federal University of Paraná, Brazil. \\Research Grant: REUNI (Projeto de Reestruturação e Expansão das Universidades Federais), Ministério da Educação e Cultura (MEC).}
%\vspace{1mm}

\section{Scholarships \& Fellowships}
\cvitem{2012--2018}{\emph{R.W. Church Scholarship}. Awarded by Department of Philosophy, UC Santa Barbara.}
%\vspace{1mm}

\cvitem{2012}{Graduate USAP Fellowship. Awarded by UC Santa Barbara.}

\cvitem{2008--2009}{Coordenação de Aperfeiçoamento Pessoal de Nível Superior (CAPES). Awarded by Federal University of Rio de Janeiro.}
\vspace{1mm}



\section{Teaching}
\cvitem{}{\faArrowCircleODown: Lower Division class, \faArrowCircleUp: Upper Division class, {\footnotesize\faArrows} postgraduate course}
\vspace{2mm} 

\subsection{Instructor}

\cvitem{CT Seminar in Philosophy of Mind \textsuperscript{\faArrowCircleODown}}{Spring 2020, Ashoka University, India.}
\vspace{.62mm}

\cvitem{Critical Thinking \textsuperscript{\faArrowCircleODown}}{Summer 2016, UC Santa Barbara, US.}
\vspace{.62mm}

\cvitem{Intro to Philosophy \textsuperscript{\faArrowCircleODown}}{Spring 2021, Ashoka University, India.}
\vspace{.62mm}

\cvitem{Logic \textsuperscript{\faArrowCircleUp}}{Summer 2011--2012, Federal University of Paraná, Brazil. %-- This teaching assignment was part of a program to prevent undergrads to drop phil classes
}
\vspace{.62mm}

\cvitem{Philosophy of Action \textsuperscript{\faArrowCircleUp\ \faArrows}}{Spring 2021, Ashoka University, India.}
\vspace{.62mm}

\cvitem{Philosophy of Language \textsuperscript{\faArrowCircleUp\ \faArrows}}{Spring 2020, Ashoka University, India.}

\vspace{2mm}

\subsection{Invited Lecturer}
\cvitem{Contemporary Philosophy II \textsuperscript{\faArrowCircleUp}}{Summer 2009, Federal University of Rio de Janeiro, Brazil.}
\vspace{1mm}


\subsection{Teaching Assistant (UC Santa Barbara)}


\cvitem{Critical Thinking \textsuperscript{\faArrowCircleODown}}{Spring 2018; Winter \& Spring 2017; Fall, Spring \& Winter 2015, Spring 2014.}
\vspace{1mm}

\cvitem{Ethics \textsuperscript{\faArrowCircleUp}}{Fall 2016.}
\vspace{1mm}

\cvitem{Epistemology \textsuperscript{\faArrowCircleUp}}{Fall 2017.}
\vspace{1mm}

\cvitem{History of Philosophy \textsuperscript{\faArrowCircleODown}}{Spring 2016.}
\vspace{1mm}

\cvitem{Intro to Ethics \textsuperscript{\faArrowCircleODown}}{Winter 2017.}
\vspace{1mm}

\cvitem{Intro to Philosophy \textsuperscript{\faArrowCircleODown}}{Fall \& Winter 2014.}
\vspace{1mm}

\cvitem{Logic \textsuperscript{\faArrowCircleUp\ \faArrows}}{Winter 2016, Fall 2013.}
\vspace{1mm}

\cvitem{Metaphysics \textsuperscript{\faArrowCircleUp}}{Summer 2014.}
\vspace{1mm}

\cvitem{Philosophy of Mind \textsuperscript{\faArrowCircleUp}}{Summer 2015.}
\vspace{1mm}


\vspace{2mm}

\subsection{Teaching Assistant (Federal University of Paraná)}
\cvitem{Logic \textsuperscript{\faArrowCircleUp}}{Spring 2007.}
\vspace{2mm}


\subsection{Reader (UC Santa Barbara)}
\cvitem{Philosophical Logic \textsuperscript{\faArrowCircleUp\ \faArrows}}{Fall 2014.}
\vspace{1mm}

\cvitem{Descartes \textsuperscript{\faArrowCircleUp}}{Summer 2014.}
\vspace{1mm}


\section{Pedagogical \& Diversity Training}

\cvitem{Summer 2020}{Foundations of Diversity and Inclusion at Work TeachOut, University of Virginia, Darden School of Business. Online, Coursera}
\vspace{1mm}

\cvitem{Summer 2016}{Summer Teaching Institute For Associates.}
\vspace{1mm}

\cvitem{Summer 2016}{Supervised Teaching Seminar in Philosophy.}
\vspace{1mm}

\cvitem{Fall 2013}{Teaching Assistant Training.}
\vspace{1mm}



%\newpage % for A4paper
\section{Professional Service}

\cvitem{Reviewer}{\emph{American Philosophical Quarterly}}
\vspace{1mm}
\cvitem{2018}{`Salmon, Schiffer, and Frege’s Constraint', Pacific APA, San Diego, US. Chair.}
\vspace{1mm}
\cvitem{2014--2016}{MAP: Minorities and Philosophy Primary Event Coordinator.}
{} % City
{} % \textit{Grade}
{%helped write the constitution of MAP chapter at UC Santa Barbara
}  % arguments 3 to 6 can be left empty
\vspace{1mm}

\cvitem{2012--2018}{Disabled Student Program (DSP) notetaker.}
\vspace{1mm}

\cvitem{2012--2015}{Volunteer to assist international students to file tax return.}
\vspace{1mm}






%\section{Languages}
%\originalcvitem{}{Portuguese (native), English (fluent), German (beginner), Spanish (beginner reading), Ancient Greek (beginner reading).}
%\vspace{1mm}



\section{Languages}
\originalcvitem{}{Portuguese (native), English (fluent), German (beginner), Spanish (beginner reading), Ancient Greek (beginner reading).}
\vspace{1mm}

%\section{Programming Skills}
%\originalcvitem{}{Python, C\raisebox{.2\height}{++}, CSS, HTML (beginner).}
%\vspace{1mm}

\section{References}
\begin{cvcolumns}
  \cvcolumn{}{\textbf{Brendan Balcerak-Jackson} \\ Department of Philosophy\\ University of Miami\\ bbalcerakjackson@gmail.com}
  \cvcolumn{}{\textbf{Kevin Falvey} \\ Department of Philosophy \\ UC Santa Barbara\\  falvey@philosophy.ucsb.edu}
  \cvcolumn{}{\textbf{Daniel Korman} \\ Department of Philosophy\\ UC Santa Barbara\\ dkorman@ucsb.edu}
  \end{cvcolumns}

  \vspace{.5cm}
\begin{cvcolumns}
  \cvcolumn{}{\textbf{François Recanati} \\ Department of Philosophy\\ Institut Jean Nicod\\ frecanati@gmail.com}
  \cvcolumn{}{\textbf{Nathan Salmon} \\ Department of Philosophy\\ UC Santa Barbara\\ nsalmon@ucsb.edu}
  \cvcolumn{}{\textbf{Stephan Torre} \\ Department of Philosophy \\ University of Aberdeen\\ stephan.torre@abdn.ac.uk}
\end{cvcolumns}

\vspace{.5cm}
\begin{cvcolumns}
  \cvcolumn{}{\textbf{Rowland Stout} \\ School of Philosophy\\ University College Dublin\\ rowland.stout@ucd.ie}
\end{cvcolumns}
\end{document}



\clearpage
%\pagestyle{fancyplain}
%\fancyhf{} % sets both header and footer to nothing
%\lfoot{\small J.F.Lima -- Dissertation Summary}
%\fancyfoot[R]{\parbox[b]{\pagenumberwidth}{\color{color2}\pagenumberfont\strut\thepage/\pageref{lastpage}}}

\begin{center}
\textbf{Dissertation Summary}

A Millian Heir Accepts the Wages of \emph{Sinn}
\end{center}

\looseness=-1
Propositions have been traditionally taken to play different roles in philosophy of language, most prominently as the meaning of (utterances of) sentences, what determines their truth-value, and the content of cognitive attitudes, like beliefs, desires, etc. In this dissertation I challenge this view. More specifically, I argue that the semantic and cognitive content of proper names are different contents, and offer an alternative theory about their relation.
\vspace{.25cm}

\looseness=-1
In Chapter 2 I survey the literature and list the theoretical roles propositions have taken to play, and explain pros and cons of two of the most well develop theories of the meaning of proper names, Millian and (a version of) Fregean Theories. 
Millian Theories, according to which the meaning of a proper names is only its referent, explain well intuitions related the meaning and truth-value of (simple) sentences with proper names in the subject position but fail to offer a suitable candidate for the content of cognitive attitudes. In contrast, Fregean Theories, according to which the meaning of a proper name is a mode of presentation of its referent, are unable to account for the meaning and truth-value of (simple) sentences with proper names, but correctly capture aspects of their cognitive content. 
At this point, a first motivation to rethink the relationship between semantic and cognitive content of names emerges, as it becomes clear that Millian and Fregean Theories have \emph{prima facie} claim over mutually exclusive intuitions. 
\vspace{.25cm}

\looseness=-1
In chapter 3 I argue that we have not been offered good reasons that semantic and cognitive content are the same. Most common arguments rely on what I argue the false claim that we cannot explain the validity of certain inferences and the truth-conditions of belief ascriptions unless they are the same content. 
\vspace{.25cm}

\looseness=-1
In Chapter 4 I develop a theory of the semantic and cognitive content of proper names which treats them as different contents. I hold Millianism for the semantic content of names, but a version of Fregeanism for their cognitive content. I further argue that we should accommodate the relevance of the cognitive content to the truth-conditions of beliefs ascriptions, such as (1) `Lois believes that Superman flies', as a parameters of evaluation determined by the context of utterance (broadly understood), and not as part of their content.  
As a result, a new account of puzzles about belief and belief ascriptions -- like Frege's Puzzle, Mates's Problem and Kripke's Puzzle about belief -- emerges. The intuitive difference in truth-value of pair of sentences like (1) and (2) `Lois believes that Clark Kent flies' is explained by the fact that they are evaluated with respect to different parameters. 
Exactly which parameter they are evaluated against depends on context of utterance, but to illustrate, in the context of a discussion of Frege's Puzzle, I argue that (1) is typically evaluated with respect to \emph{the cognitive content Lois associated with the sentence `Superman flies'}, whereas (2) is evaluated with respect to \emph{the cognitive content Lois associated with the sentence `Clark Kent flies'}, which are different and only one of which is in Lois's belief box, so to speak. 
\vspace{.25cm}

\looseness=-1
My view improves over current Millian accounts because, among other things, it does not rely on claims about pragmatics of belief ascriptions, hidden indexicality, or other hypotheses that hold that the belief-relation a ternary relation. Besides, it offers a way of ascribing beliefs to non-human animals without assuming they have conceptual abilities like ours.

\end{document}





%-----       letter       ---------------------------------------------------------
% recipient data
\recipient{Company Recruitment team}{Company, Inc.\\123 somestreet\\some city}
\date{January 01, 1984}
\opening{Dear Sir or Madam,}
\closing{Yours faithfully,}
%\enclosure[Attached]{curriculum vit\ae{}}          % use an optional argument to use a string other than "Enclosure", or redefine \enclname
\makelettertitle

Lorem ipsum dolor sit amet, consectetur adipiscing elit. Duis ullamcorper neque sit amet lectus facilisis sed luctus nisl iaculis. Vivamus at neque arcu, sed tempor quam. Curabitur pharetra tincidunt tincidunt. Morbi volutpat feugiat mauris, quis tempor neque vehicula volutpat. Duis tristique justo vel massa fermentum accumsan. Mauris ante elit, feugiat vestibulum tempor eget, eleifend ac ipsum. Donec scelerisque lobortis ipsum eu vestibulum. Pellentesque vel massa at felis accumsan rhoncus.

Suspendisse commodo, massa eu congue tincidunt, elit mauris pellentesque orci, cursus tempor odio nisl euismod augue. Aliquam adipiscing nibh ut odio sodales et pulvinar tortor laoreet. Mauris a accumsan ligula. Class aptent taciti sociosqu ad litora torquent per conubia nostra, per inceptos himenaeos. Suspendisse vulputate sem vehicula ipsum varius nec tempus dui dapibus. Phasellus et est urna, ut auctor erat. Sed tincidunt odio id odio aliquam mattis. Donec sapien nulla, feugiat eget adipiscing sit amet, lacinia ut dolor. Phasellus tincidunt, leo a fringilla consectetur, felis diam aliquam urna, vitae aliquet lectus orci nec velit. Vivamus dapibus varius blandit.

Duis sit amet magna ante, at sodales diam. Aenean consectetur porta risus et sagittis. Ut interdum, enim varius pellentesque tincidunt, magna libero sodales tortor, ut fermentum nunc metus a ante. Vivamus odio leo, tincidunt eu luctus ut, sollicitudin sit amet metus. Nunc sed orci lectus. Ut sodales magna sed velit volutpat sit amet pulvinar diam venenatis.

Albert Einstein discovered that $e=mc^2$ in 1905.

\[ e=\lim_{n \to \infty} \left(1+\frac{1}{n}\right)^n \]

\makeletterclosing

%\clearpage\end{CJK*}                              % if you are typesetting your resume in Chinese using CJK; the \clearpage is required for fancyhdr to work correctly with CJK, though it kills the page numbering by making \lastpage undefined


-----------------------------------
\section{Computer skills}
\cvdoubleitem{category 1}{XXX, YYY, ZZZ}{category 4}{XXX, YYY, ZZZ}
\cvdoubleitem{category 2}{XXX, YYY, ZZZ}{category 5}{XXX, YYY, ZZZ}
\cvdoubleitem{category 3}{XXX, YYY, ZZZ}{category 6}{XXX, YYY, ZZZ}

\section{Interests}
\cvitem{hobby 1}{Description}
\cvitem{hobby 2}{Description}
\cvitem{hobby 3}{Description}

\section{Extra 1}
\cvlistitem{Item 1}
\cvlistitem{Item 2}
\cvlistitem{Item 3. This item is particularly long and therefore normally spans over several lines. Did you notice the indentation when the line wraps?}

\section{Extra 2}
\cvlistdoubleitem{Item 1}{Item 4}
\cvlistdoubleitem{Item 2}{Item 5\cite{book1}}
\cvlistdoubleitem{Item 3}{Item 6. Like item 3 in the single column list before, this item is particularly long to wrap over several lines.}


-----------------
% Publications from a BibTeX file without multibib
%  for numerical labels: \renewcommand{\bibliographyitemlabel}{\@biblabel{\arabic{enumiv}}}% CONSIDER MERGING WITH PREAMBLE PART
%  to redefine the heading string ("Publications"): \renewcommand{\refname}{Articles}
\nocite{*}
\bibliographystyle{plain}
\bibliography{publications}                        % 'publications' is the name of a BibTeX file

% Publications from a BibTeX file using the multibib package
%\section{Publications}
%\nocitebook{book1,book2}
%\bibliographystylebook{plain}
%\bibliographybook{publications}                   % 'publications' is the name of a BibTeX file
%\nocitemisc{misc1,misc2,misc3}
%\bibliographystylemisc{plain}
%\bibliographymisc{publications}                   % 'publications' is the name of a BibTeX file
------------